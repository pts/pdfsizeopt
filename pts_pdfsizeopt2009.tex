% by pts@fazekas.hu at Sat Jul 11 12:01:05 CEST 2009

\documentclass{article}
%\usepackage[latin2]{inputenc}% !! check
\usepackage{t1enc}
\usepackage{lmodern}
\usepackage[english]{babel}

\def\cmd{\textsf}
\def\pkg{\textsf}


\begin{document}

\section{What is PDF?}

PDF is a popular document file format designed for printing and on-screen
viewing. PDF faithfully preserves the design elements of the document, such
as fonts, line breaks, page breaks, exact spacing, text layout, vector
graphics and image resolution. Thus the author of a PDF document has
precise control over the document's appearance -- no matter what operating
system or renderer software is used for viewing or printing the PDF. From
the viewer's prespecive, a PDF document is a sequence of rectangular pages
containing text, vector graphics and images. In addition to that, some
rectangular page regions can be marked as hyperlinks, and Unicode
annotations can be added as well to the regions, so text can be copy-pasted
from the documents. (Usually the copy-paste yields only a sequence of
characters, with all formatting and positioning lost. Depending on the
software and the annotation, the bold and italics properties can be
preserved.) A table of contents can be added as well, which has a tree
structure, and each node of the tree consists of an unformatted caption and
a hyperlink within the document.

Additional features of PDF include forms (the user fills some fields with
data, clicks on the submit button, and the data is sent to a server in an
HTTP request), event handlers in JavaScript, embedded multimedia files,
encryption and access protection.

\section{How to create PDF}

Since PDF doesn't contain semantic information about the document (such as
in which order the document should be read, which regions are titles, how
are the tables built and how are the charts generated), word processors and
typesetting systems usually can export to PDF, but they have their own file
format which preserves semantics. PDF is usually not involved while the
author is composing (or typeseting) the document, but once a version of a
document is ready, a PDF can be exported and distributed. Should the author
distribute the document in the native file format of the word processor, he
might risk that the document doesn't get rendered as he intended, due to
software version differences or because slightly different fonts are
installed on the rendering computer, or the page layout settings in the word
processor are different.

Most word processors and drawing programs and image editors support
exporting as PDF. It is also possible to generate a PDF even if the software
doesn't have a PDF export feature. For example, it may be possible to
install a printer driver, which generates PDF instead of sending the
document to a real printer. (For example, on Windows, PDFCreator
\cite{pdfcreator} is such an open-source driver.) Some old programs can emit
PostScript, but not PDF. The \cmd{ps2pdf} \cite{ps2df} tool (part of
Ghostcript) can be used to convert the PostScript to PDF.

For \TeX{} documents there are several options for PDF generation. One
traditional approach is \TeX{} source $\to$ DVI $\to$ PostScript $\to$ PDF,
using \cmd{dvips} \cite{dvips} for creating the PostScript file, and
\cmd{ps2pdf} for creating the PDF file. One can convert directly from DVI to
PDF, using e.g.\ \cmd{dvipdfm} \cite{dvipdfm}. The single-step approach is
processing the \TeX{} source with pdf\TeX{} \cite{pdftex}, which generates
PDF without any intermediate files. We recommend using pdf\TeX{} whenever
possible, because it creates a small PDF file, it is fast, and it supports
the most PDF features without compromises. For example, the other approaches
cannot break the line in the middle of a hyperlink, but pdf\TeX{} can even
hyphenate the link text automatically.

Unfortunately, some graphics packages (such as \pkg{psfrag} and
\pkg{pstricks}) require a PostScript backend, and pdf\TeX{} doesn't provide
that. We recommend typesetting new graphics with TikZ \cite{tikz} if possible
(which fully supports pdf\TeX{}), and converting other graphics to PDF,
possibly typesetting label captions using \TeX{}, as outlined in
\cite{psfrag-in-pdf}. Inkscape users can use \pkg{textext} \cite{textext}
within Inkscape to make \TeX{} typeset the captions.

With pdf\TeX{}, it is possible to reuse existing PDF files by embedding
parts of them to the document. The \texttt{\string\includegraphics} command
of the standard \pkg{graphicx} \LaTeX{}-package accept a PDF as the image
file. In this case, the first page of the specified PDF will be used as a
rectangular image. For multi-page inclusions, the \pkg{pdfpages}
\LaTeX{}-package can be used. Please note that due to a limitation in
pdf\TeX{}, hyperlinks and outlines (table of contents) in the embedded PDF
will be lost.

\section{Motivation}

Our goal is to reduce the file size of PDF documents, focusing on PDF files
created from \TeX{} documents. Having smaller PDF files reduces download
times, web hosting costs and storage costs as well. Although there is no
urgent need for reducing PDF storage costs is for personal use (since hard
drives in modern PCs are large enough), storage costs are significant for
publishing houses, print shops, e-book stores and hosting services,
libraries and archives \cite{multivalent-article}.
Usually lots of copies and backups are made of PDF
files originating from such places, saving 20\% of the file size right after
generating the PDF would save 20\% of all future costs associated with the
file.

Although e-book readers can store lots of documents (e.g.\ a 4\,GB e-book
reader can store 800 PDF books of 5\,MB average reasonable file size), they
get full quickly if we don't pay attention to efficient PDF generation. One
can easily get a PDF file 5 times larger than reasonable by using
inefficient software to generate it, or not setting the export settings
properly. Upgrading or changing the generator software is not always
feasible. A PDF recompressor becomes useful in these cases.

It is not our goal to propose alternative file formats, which support a more
compact document representation or more agressive compression than PDF. An
example for such an approach is the Multivalent \emph{compact} file format
\cite{multivalent-compact}, which can be generated from a PDF using the
Multivalent tools, and it can be converted back to a regular PDF with these
tools as well. One of the merits of PDF is that it has a definitive, widely
accepted and implemented, freely available
specification \cite{pdfref} (and version 1.7 is
even ISO standard \cite{pdf-iso}), so if 20 years later we want to
print the document we've create today, then a well-documented and
standardized format such as PDF is a natural choice. (Please note, however,
that the PDF specification is not self-contained, it refers to other
specifications, e.g.\ for some compression algorithms and font formats.)

An alternative document file format is DjVu (\cite{djvu}), whose most
important limitation compared to PDF is that it doesn't support vector
graphics. Due to the sophisticated image segmentation and compression, a
600\,DPI DjVu file is usually smaller than the coressponding PDF document
containing text with embedded vector fonts and vector graphics. Since the
DjVu file format uses very different technologies than PDF, one can archive
both the PDF and the DjVu version of the same document, in case a decent
renderer won't be available for one of the formats decades later.

\section{References}

!! \cite all

@psfrag-in-pdf{
  url=http://brunoj.wordpress.com/2007/07/12/producing-high-quality-eps-and-pdf-diagrams-for-mathematics/
}

@iso-pdf{
PDF 1.7 is ISO 32000
meta-link: http://www.theinquirer.net/inquirer/news/1030411/pdf-approved-iso-32000
}

\end{document}
